%% LyX 2.1.4 created this file.  For more info, see http://www.lyx.org/.
%% Do not edit unless you really know what you are doing.
\documentclass[english]{paper}
\usepackage[T1]{fontenc}
\usepackage[latin9]{inputenc}
\usepackage[letterpaper]{geometry}
\geometry{verbose,tmargin=1in,bmargin=1in,lmargin=1in,rmargin=1in}
\pagestyle{plain}
\usepackage{babel}
\usepackage[unicode=true]
 {hyperref}

\makeatletter

%%%%%%%%%%%%%%%%%%%%%%%%%%%%%% LyX specific LaTeX commands.
\newcommand{\noun}[1]{\textsc{#1}}

\makeatother

\begin{document}

\title{Kindle KOA2 (Kindle Oasis, 9th generation, 2017)}

\maketitle
Not released for distribution - Experimental use only.


\section{Required}

Required to retain control of the Kindle firmware version and make
the \emph{Update Your Kindle} (UYK) menu item available.


\subsection{Prevent OTA Update}

In the topmost visible level of USB storage, add the OTA blocking
directory name.\\
The blocking directory name is:\\
\emph{update.bin.tmp.partial}\\
Use that exact name, no additional extension.


\subsection{Register the Kindle}

With some firmware versions, you can not access the \emph{Update Your
Kindle} (UYK) menu unless the device is registered. All of the KOA2
experimental {*}.bin packages install with UYK. Using MrPI is not
required.




\section{Recommended}

The following changes are all in support of recovering the Kindle
should start-up problems occur. Each can prevent the export of USB
storage until a touch screen action is taken. But most start-up problems
occur before the touch screen is up and working.
\begin{itemize}
\item Remove any passcode protection.
\item Remove any parental controls.
\item Have special offers removed.
\end{itemize}

\section{\label{sec:Check-current-firmware}Check current firmware}

Test if the Kindle can be jail broken with the current firmware.

Note: Firmware 5.9.2 can not (yet) be jail broken.
\begin{itemize}
\item Download \href{https://www.adrive.com/public/U3AVtU/main-htmlviewer.tar.gz}{Device Jail Break}
\end{itemize}
Do not let your PC open that archive or otherwise convert it to a
\emph{safe} archive. The archive content structure makes it a tar
bomb, it is suppose to be a tar bomb, that is what makes it work.
Owners of \noun{MacOS} systems should be aware that the default settings
will open and convert this archive to a \emph{safe} archive, which
prevents it from working as intended.
\begin{enumerate}
\item Place in topmost visible level of USB storage.
\item Safely remove the USB cable.
\item In the search bar of the home screen, enter:\\
;installHtml\\
The semi-colon is part of the command and the command is case sensitive.
\item \label{enu:Did-you-have}Did you have a \emph{JailBroken} document
appear on your Kindle?

\begin{description}
\item [{Yes~it~did~appear:}] The device jail break is now installed,
you can skip the installation of a \emph{Factory Use Only }firmware
build. Continue at section \ref{sec:Install-Device-Jail}.
\item [{No~it~did~not~appear:}] To be certain, use your PC to look
for it in the /documents folder. In this case you will have to install
a \emph{Factory Use Only} firmware build as described below.
\end{description}
\end{enumerate}

\section{Install a \emph{Factory Use Only} firmware build}

If the \emph{JailBroken }document showed up on your Kindle when you
did step \ref{enu:Did-you-have} of section \ref{sec:Check-current-firmware}
above, you may skip this section.


\subsection{Replace your \emph{Customer Use} firmware build with a \emph{Factory
Use Only} firmware.}

In this case, the firmware version numbers before and after this step
are not significant.

The use of the word \emph{Update} means the label on the button in
the settings menu, it does not describe the action performed.
\begin{enumerate}
\item Download the firmware image that was intended to only be used on the
factory production line: \href{https://www.adrive.com/public/RyfAaK/update_kindle_oasis_9th_factory_5.9.0.6.bin}{Factory-5.9.0.6}.
\item Place in the topmost visible level of USB storage.
\item Remove USB cable.
\item Home -> Menu -> Settings -> Menu -> Update (Your Kindle - UYK)
\item Wait.\\
Do not panic. These factory image package may take as long as 5 minutes
to install.
\end{enumerate}

\section{\label{sec:Install-Device-Jail}Install Device Jail Break}

Note: Customer firmware 5.9.2 can not (yet) be jail broken.
\begin{itemize}
\item Download \href{https://www.adrive.com/public/U3AVtU/main-htmlviewer.tar.gz}{Device Jail Break}
\end{itemize}
Do not let your PC open that archive or otherwise convert it to a
\emph{safe} archive. The archive content structure makes it a tar
bomb, it is suppose to be a tar bomb, that is what makes it work.
Owners of \noun{MacOS} systems should be aware that the default settings
will open and convert this archive to a \emph{safe} archive, which
prevents it from working as intended.
\begin{enumerate}
\item Place in topmost visible level of USB storage.
\item Safely remove the USB cable.
\item In the search bar of the home screen, enter:\\
;installHtml\\
The semi-colon is part of the command and the command is case sensitive.
\item \label{enu:Did-you-have-1}Did you have a \emph{JailBroken} document
appear on your Kindle?\end{enumerate}
\begin{description}
\item [{Yes~it~did~appear:}] The device jail break is now installed,
continue with the next section.
\item [{No~it~did~not~appear:}] To be certain, use your PC to look
for it in the /documents folder. In this case something unexpected
has happened, contact the developers.
\end{description}

\section{Install the Jailbreak Survial Code}

This package has also been known as the \emph{hotfix} package.

This installs both the Bridge Code that auto-reinstalls the device
jailbreak and it install the application keys required to run some
of the add-in applications.

Note: This step may have to be repeated after every change in registration
status.
\begin{enumerate}
\item Download the \href{https://www.adrive.com/public/xpYGey/Update_jailbreak_hotfix_1.15_koa2.bin}{Jailbreak Survial Code}
\item Place in the topmost visible level of USB storage.
\item Safely remove the USB cable.
\item Home -> Menu -> Settings -> Menu -> Update (Your Kindle - UYK)
\item Watch the screen while waiting.\\
The wait should only be that involved in any package installation.
\end{enumerate}
\bigskip{}



\section{Notice}

Any other \emph{update\_{*}.bin} name format package used \noun{must}
be re-installed after an Amazon update.

Only the \emph{Device Jail Break} and components of the \emph{Jail
Break Survival Code} are auto-reinstalled.

In the usual case, KUAL extensions do not need to be re-installed,
but if anything seems to be broken, re-install it.

\bigskip{}



\section{Experimental KOA2 Packages}

All of these packages are built to install using the UYK (Update Your
Kindle) menu entry.

The Mobileread Package Installer (MrPI) is not required for installing
these packages.

All of these are the package only, refer to the original release post
for directions and/or the directions included in the corresponding
archives in the listings at \href{https://www.mobileread.com/forums/showthread.php?t=225030}{NiLuJe's Snapshots thread}


\subsection{Required}

With very few exceptions, these items are required.
\begin{description}
\item [{KUAL~Launcher}] Provides the application launcher menu.

\begin{itemize}
\item Release~post: \href{https://www.mobileread.com/forums/showthread.php?t=203326}{Kindle Unified Application Launcher}
\item Install: \href{https://www.adrive.com/public/9g9eX9/Update_KUALBooklet_v2.7_koa2_nomax_install.bin}{KUAL Booklet install}
\item Status: Tested, Some Undescribed Problems Reported

\begin{itemize}
\item coplate: ``Works for me.''
\end{itemize}
\item Uninstall: \href{https://www.adrive.com/public/3VEpNg/Update_KUALBooklet_v2.7_koa2_nomax_uninstall.bin}{KUAL Booklet uninstall}
\item Status: Not Tested
\end{itemize}
\end{description}

\subsection{Suggested}
\begin{description}
\item [{Rescue~Pack}] Adds SSH server to Diags system and restores the
detection of ENABLE\_DIAGS in the topmost level of USB storage.

\begin{description}
\item [{Release~post:}] \href{https://www.mobileread.com/forums/showthread.php?t=195670}{Rescue Pack for Paperwhite and Touch}
\item [{Install:}] \href{https://www.adrive.com/public/Hkd8VF/Update_rp_20131220.N_install_koa2_nomax.bin}{Rescue Pack install}
\item [{Status:}] Tested, Broken, Do Not Use
\item [{Uninstall:}] Never provided.
\item [{Status:}] N/A
\end{description}
\item [{Coward's~Rescue~Pack}] Optional add-on to the Rescue Pack. Provides
control of Rescue Pack by detecting USB cable connection.

\begin{description}
\item [{Release~post:}] \href{https://www.mobileread.com/forums/showthread.php?t=232507}{Coward's Rescue Pack, a Rescue Pack add-on}
\item [{Install:}] \href{https://www.adrive.com/public/g6WJSG/Update_crp_2.N_install_koa2_nomax.bin}{Coward's Rescue Pack install}
\item [{Status:}] Not tested.
\item [{Uninstall:}] \href{https://www.adrive.com/public/bSbTA2/Update_crp_2.N_uninstall_koa2_nomax.bin}{Coward's Rescue Pack uninstall}
\item [{Status:}] Not tested.
\end{description}
\end{description}

\subsection{Available}
\begin{description}
\item [{USB~Networking}] Includes both ssh and telnet servers. Many bonus
items also included. Note: The Amazon/Lab126 version may be included
in some Factory Use firmware builds. Detail undetermined at this time.

\begin{description}
\item [{Release~post:}] \href{https://www.mobileread.com/forums/showthread.php?t=186645}{USB Networking}
\item [{Install:}] \href{https://www.adrive.com/public/DhYefd/Update_usbnet_0.21.N_install_mx7_koa2_nomax.bin}{USB Networking install}
\item [{Status:}] Tested, working with some problems

\begin{description}
\item [{coplate:}] ``stopping usbnetworking (with KUAL) crashes Kindle,
requires a reboot''
\end{description}
\item [{Uninstall:}] \href{https://www.adrive.com/public/JVtudU/Update_usbnet_0.21.N_uninstall_koa2_nomax.bin]}{USB Networking uninstall}
\item [{Status:}] Not tested.
\end{description}
\end{description}

\subsection{Packaging Tools}

Tools and utilities for the examination and maintenance of \emph{update\_{*}.bin}
packages.
\begin{description}
\item [{KindleTool~Binaries}] KindleTool both creates new and opens existing
\emph{update\_{*}.bin} format packages. Both those of Amazon/Lab126
and those of Mobileread.

\begin{description}
\item [{Release~post:}] https://www.mobileread.com/forums/showthread.php?t=187880
\item [{LInux~x86\_64:}] https://www.adrive.com/public/WkZAZR/KindleTool-v1.6.4.108-ge16765c-linux-x86\_64.zip
\item [{Status:}] Working, in-use
\item [{Windows~native~64bit:}] .
\item [{Status:}] Not yet available.
\end{description}
\item [{Repackaging~Script}] The script contains a table format listing
of Mobileread update\_{*}.bin packages.

\begin{description}
\item [{Script:}] \href{https://www.adrive.com/public/mZkvMb/mkpkgs.sh.zip}{mkpkgs batch script}
\item [{Status:}] Working, in use.
\end{description}
\end{description}

\subsection{Historical Interest Only}
\begin{description}
\item [{Old~factory~images}] Not intended to be used, just things lying
around and posted as possible information. Any or all of these may
be corrupt.

\begin{description}
\item [{Main~system:}] \href{https://www.adrive.com/public/DTYvtt/update-J9.0.5.1-zelda_cognac-007.bin}{update\_{}main\_{}5.9.0.5.1-007}
\item [{Main~system:}] \href{https://www.adrive.com/public/uqYyFZ/update-J9.0.5.1-zelda_cognac-008.bin}{update\_{}main\_{}5.9.0.5.1-008}
\item [{Diag\_system:}] \href{https://www.adrive.com/public/SaRXg2/update-J9-zelda_cognac-001.059-diags.bin}{update\_{}diags\_{}001.059}
\end{description}
\end{description}

\subsection{KUAL Extensions}

KUAL extensions typically are not provided in an update\_{*}.bin name
format package. The are distributed in archives to be un-archived
to the topmost visible level of USB storage.
\begin{description}
\item [{Kindle~Terminal}] Provides an interactive, command line, terminal
with on-screen keyboard.

\begin{description}
\item [{Archive:}] \href{https://www.adrive.com/public/P46YNY/kterm-kindle-2.4.zip}{kTerm}
\item [{Status:}] Not tested.\end{description}
\end{description}

\end{document}
