%% LyX 2.1.4 created this file.  For more info, see http://www.lyx.org/.
%% Do not edit unless you really know what you are doing.
\documentclass[english]{paper}
\usepackage[T1]{fontenc}
\usepackage[latin9]{inputenc}
\usepackage[letterpaper]{geometry}
\geometry{verbose,tmargin=1in,bmargin=1in,lmargin=1in,rmargin=1in}
\pagestyle{plain}
\usepackage{babel}
\usepackage[unicode=true]
 {hyperref}

\makeatletter

%%%%%%%%%%%%%%%%%%%%%%%%%%%%%% LyX specific LaTeX commands.
\newcommand{\noun}[1]{\textsc{#1}}

\makeatother

\begin{document}

\title{Kindle KOA2 (Kindle Oasis, 9th generation, 2017)}

\maketitle
\tableofcontents{}


\section{Firmware versions}

The process described here applies to only two firmware versions.

If the Kindle has either of these versions installed when first taken
out of the box:
\begin{itemize}
\item 5.9.0.5.1
\item 5.9.2.0.1
\end{itemize}
Then this process should work.


\section{Highly recommended}

To retain control of the Kindle firmware version, in the topmost visible
level of USB storage, add the OTA blocking directory name. The blocking
directory name is:\\
\emph{update.bin.tmp.partial}\\
Use that exact name, no additional extension.

There is no guarantee that this will block the download of an OTA
update. 

There have not been any firmware versions reported (yet) that can
over-ride this blocking. 

It is possible that Amazon/Lab126 will change this situation at any
time, without notice.


\section{Recommended}

The following changes are all in support of recovering the Kindle
should start-up problems occur.

Each can prevent the export of USB storage until a touch screen action
is taken. Most start-up problems occur before the touch screen is
up and working.
\begin{itemize}
\item Register the Kindle. 
\item Remove any passcode protection.
\item Remove any parental controls.
\item Have special offers removed.
\end{itemize}

\section{Install a \emph{Factory Use Only} firmware build}

The firmware version numbers before and after this step are not significant.

The use of the word \emph{Update} means the label on the button in
the settings menu, it does not describe the action performed.
\begin{enumerate}
\item Download the firmware image that was intended to only be used on the
factory production line: \href{https://www.adrive.com/public/RyfAaK/update_kindle_oasis_9th_factory_5.9.0.6.bin}{Factory-5.9.0.6}.
\item md5sum: 18624db8c1838ec2b5b8bfb3406ac041
\item Place in the topmost visible level of USB storage.
\item Remove USB cable.
\item Home -> Menu -> Settings -> Menu -> Update (Your Kindle - UYK)
\item Wait.\\
Do not panic. This factory image package may take as long as 5 minutes
to install.
\end{enumerate}

\section{Install Device Jail Break}

Customer firmware versions other than the two listed above can not
(yet) be jail broken.
\begin{itemize}
\item Download the \href{https://www.adrive.com/public/U3AVtU/main-htmlviewer.tar.gz}{Device Jail Break}
\end{itemize}
Do not let your PC open the archive or otherwise convert it to a \emph{safe}
archive. The archive content structure makes it a tar bomb, it is
suppose to be a tar bomb, that is what makes it work. Owners of \noun{MacOS}
systems should be aware that the default settings will open and convert
this archive to a \emph{safe} archive, which prevents it from working
as intended.
\begin{enumerate}
\item md5sum: 8d4ef0528bc1d72576b890a72840780a\\
This value will match if the download was without error and if your
PC has not tried to safely re-pack it.
\item Place in topmost visible level of USB storage.
\item Safely remove the USB cable.
\item In the search bar of the home screen, enter:\\
;installHtml\\
The semi-colon is part of the command and the command is case sensitive.
\item Did you have a \emph{JailBroken} document appear on your Kindle?\end{enumerate}
\begin{itemize}
\item \noun{Yes it did appear:} The device jail break is now installed,
continue with the next section.
\item \noun{No it did not appear:} To be certain, use your PC to look for
it in the /documents folder. In this case something unexpected has
happened, contact the developers.
\end{itemize}

\section{Install the Jailbreak Survival Code}

This package has also been known as the \emph{hotfix} package.

This installs both the Bridge Code that auto-reinstalls the device
jailbreak and it installs the application keys required to run some
of the add-in applications.
\begin{itemize}
\item \textsf{Note:} This step may have to be repeated after every change
in registration status.\end{itemize}
\begin{enumerate}
\item Download the \href{https://www.adrive.com/public/Srz2x2/Update_jailbreak_hotfix_1.15_koa2.bin}{Jailbreak Survival Code}
\item md5sum: 19857c59d350470afff27f4249be8bac
\item Place in the topmost visible level of USB storage.
\item Safely remove the USB cable.
\item Home -> Menu -> Settings -> Menu -> Update (Your Kindle - UYK)
\item Watch the screen while waiting.\\
The wait should only be that involved in any package installation.
\end{enumerate}
\bigskip{}



\section{Notice}

Any other \emph{update\_{*}.bin} name format package used \noun{must}
be re-installed after an Amazon update.

Only the \emph{Device Jail Break} and components of the \emph{Jail
Break Survival Code} are auto-reinstalled.

In the usual case, KUAL extensions do not need to be re-installed,
but if anything seems to be broken, re-install it.

\bigskip{}



\section{Experimental KOA2 Packages}

All of these packages are built to install using the UYK (Update Your
Kindle) menu entry.

The Mobileread Package Installer (MrPI) is not required for installing
these packages.


\subsection{Available}
\begin{itemize}
\item USB~Networking Includes both ssh and telnet servers. Many bonus items
also included. Note: Without an application launcher (KUAL), you will
have to use the ;un searchbar command to toggle between USBnetworking
and USBmass storage modes.

\begin{itemize}
\item Release post: \href{https://www.mobileread.com/forums/showthread.php?t=186645}{USB Networking}
\item Install: \href{https://www.adrive.com/public/DhYefd/Update_usbnet_0.21.N_install_mx7_koa2_nomax.bin}{USB Networking install}
\item Status: Tested, working.
\item Uninstall: \href{https://www.adrive.com/public/JVtudU/Update_usbnet_0.21.N_uninstall_koa2_nomax.bin]}{USB Networking uninstall}
\item Status: Not tested.
\end{itemize}
\end{itemize}

\subsection{Packaging Tools}

Tools and utilities for the examination and maintenance of \emph{update\_{*}.bin}
packages.
\begin{itemize}
\item KindleTool Binaries KindleTool both creates new and opens existing
\emph{update\_{*}.bin} format packages. Both those of Amazon/Lab126
and those of Mobileread.

\begin{itemize}
\item Release post: \href{https://www.mobileread.com/forums/showthread.php?t=187880}{NiLuJe's KindleTool}
\item Current builds: \href{https://www.mobileread.com/forums/showthread.php?t=225030}{KindleTool for Linux, MacOS and Windows}
\item Status: Working, in-use
\end{itemize}
\item Repackaging Script The script contains a table format listing of Mobileread
update\_{*}.bin packages.

\begin{description}
\item [{Script:}] \href{https://www.adrive.com/public/mZkvMb/mkpkgs.sh.zip}{mkpkgs batch script}
\item [{Status:}] Working, in use.
\end{description}
\end{itemize}

\subsection{KUAL Extensions}

Since a usable version of KUAL does not exist, KUAL extensions must
be started using the searchbar command: ;log runme along with a custom
script in USB storage.

A collection of ``RUNME.sh'' scripts is being made in this thread:
\href{https://www.mobileread.com/forums/showthread.php?t=292382}{Collection of runme scripts}
\begin{itemize}
\item Kindle Terminal\noun{ }Provides an interactive, command line, terminal
with on-screen keyboard.

\begin{itemize}
\item Release Post:\noun{ }\href{https://www.mobileread.com/forums/showthread.php?t=179286}{Terminal Emulator for Touchscreen Kindles}
\item Public repository: \href{https://github.com/bfabiszewski/kterm}{Source code repository}
\item Archive: \href{https://www.adrive.com/public/P46YNY/kterm-kindle-2.4.zip}{kTerm}
\item md5sum: 5f943d7928d6fa206514241a7f245081
\item Status: A working runme script has been contributed.\end{itemize}
\end{itemize}

\end{document}
